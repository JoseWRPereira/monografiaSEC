%CONTEXTO: Identifica a grande area de pesquisa e sua importancia;

%LACUNA: (however:contudo, all do:todos fazem) O que precisa ser estudado nesse campo e ainda precisa ser entendido, exclarecido. Local aonde o trabalho está inserido;

%PRÓPOSITO: (this paper describes...)o que foi feito e principal objetivo do artigo;

%METODOLOGIA: Maneira geral falar dos métodos

%RESULTADOS: IMPORTANTÍSSIMO!!! Principal achado, resultado de forma bastante clara;

%CONCLUSOES: Mostrar como que os resutados contribuem para o avanço da grande aréa.

Sistemas de Controle são largamente utilizados principalmente no setor industrial e buscam uma maior eficiência de tempo e energia, mantendo a qualidade dos processos e do sistema controlado, 
% Lacuna
contudo ainda são muito complexos e de difícil implementação, necessitando de um sistema embarcado dedicado.
% Propósito
O objetivo deste estudo é mostrar uma forma alternativa de controle em malha fechada e implementar, um controle não convencional utilizando a Lógica Paraconsistente Anotada com anotação de dois valores (LPA2v), 
% Metodologia
de forma comparativa ao modelo clássico de controle, Proporcional, Integral e Derivativo (PID).
% Resultados
Partindo dos conceitos básicos da Lógica Paraconsistente, 
mostrando as principais formas de uso baseadas em 
trabalhos anteriores de diversos autores, 
foi aplicada uma variação na forma de utilização da LPA2v 
de forma a efetuar o controle de forma conveniente e adequada 
de acordo com os requisitos de desempenho propostos.
%Conclusão		
Comparativamente, o modelo proposto alinha-se ao modelo clássico, tendo parte de sua teoria adaptada para o atendimento dos pressupostos da LPA2v.
		

