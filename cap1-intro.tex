Comparar sistemas de controle mediante uma mesma planta é uma forma de 
avaliar as possibilidades e limitações de ambos os elementos de estudo, 
possibilitando uma melhor escolha no momento de planejar e executar um projeto, 
obtendo assim um ganho de tempo, 
que reflete diretamente no custo de implementação e manutenção 
além de conferir ao projeto maior possibilidade de atingir 
um melhor desempenho e uma maior confiabilidade. 

Sistemas de controle são largamente utilizados pela indústria como um todo a muitos anos, 
tendo algumas técnicas amplamente difundidas e com alto grau de maturação, 
como é o caso do controle Proporcional-Integral-Derivativo (PID), \nomenclature{PID}{Proporcional-Integral-Derivativo} %
mesmo apresentando restrições e limitações quanto a aplicação em sistemas que possuem não-linearidades, 
atrasos de transporte e/ou parâmetros variantes no tempo \cite{Ferreira2012}.


Tendo em vista que estudos de novas formas de controle não clássicas estão em curso, 
a lógica paraconsistente surge como uma promissora ferramenta para tomada de decisão em diversos campos de aplicação como 
a robótica, automação industrial, inteligência artificial, logística, controle, entre outras\cite{JoaoInacio}.
 
O presente trabalho tem como objetivo a caracterização de duas teorias e 
sua utilização em sistemas controle, 
PID e Lógica Paraconsistente Anotada de anotação com dois valores (LPA2v), 
\nomenclature{LPA2v}{Lógica Paraconsistente Anotada de anotação com dois valores}%
que é uma das formas de aplicação da Lógica Paraconsistente, 
sua implementação e posterior comparação utilizando 
uma plataforma que contemple recursos que 
possibilite uma análise de desempenho e complexidade de implementação.
%Objetivo ou Finalidade: preocupação em distinguir a característica comum ou as leis gerais que regem determinados eventos.



%%%%%%%%%%%%%%%%%%%%%%%%%%%%%%%%%%%%%%%%%%%%%%%%%%%%%%%%%%%%
\section{Formulação do Problema}
%%%%%%%%%%%%%%%%%%%%%%%%%%%%%%%%%%%%%%%%%%%%%%%%%%%%%%%%%%%%



O cenário dos dispositivos microcontrolados é cada vez maior e 
abrange uma gama de aplicações muito ampla, 
desde pequenas aplicações com dispositivos de 8-bits até 
modernos controladores de 32-bits integrados com 
hardware dedicado a processamento digital de sinais e cálculos avançados.

Algumas lógicas ainda não tiveram  uma abordagem prática de sua implementação, 
ou ainda, tais abordagens são muito escassas, 
seja com dispositivos simples ou com os mais complexos.

Assim surgiu a questão que aqui se apresenta: 
Existe a possibilidade de substituir um controlador PID por 
um controlador baseado em lógica paraconsistente, 
e obter um resultado equivalente em termos de qualidade da resposta do sistema, 
ou seja, atendendo requisitos de desempenho de um dado sistema? 

Para responder a essa questão é implementado um sistema com 
um controlador de 32-bits com núcleo ARM e \nomenclature{ARM}{\emph{Advanced RISC Machine} }\nomenclature{RISC}{\emph{Reduced Instruction Set Computer} }% 
que possui hardware dedicado, 
para a comparação entre sistema de controle PID e controle baseado em LPA2v.



%%%%%%%%%%%%%%%%%%%%%%%%%%%%%%%%%%%%%%%%%%%%%%%%%%%%%%%%%%%%
\section{Hipótese e Relevância do Trabalho}
%%%%%%%%%%%%%%%%%%%%%%%%%%%%%%%%%%%%%%%%%%%%%%%%%%%%%%%%%%%%

A lógica paraconsistente vem ganhando relevância e adeptos 
principalmente a partir do final da década de 90 do século XX, 
quando houve o Primeiro Congresso Mundial sobre Paraconsistência em 
Gent na Bélgica em 1997, 
no ano 2000 o segundo congresso realizado em São Sebastião, São Paulo e 
o terceiro em Toulouse, França em julho de 2003, 
atraindo cada vez mais pesquisadores interessados de 
diversos centros de pesquisa do mundo \cite{DecioKrause}. 
Em meados de setembro de 2016, 
aconteceu o pela primeira vez no Brasil a 
XVI Conferência Internacional de Lógica: 
\texttt{Tendências da Lógica} 
(\emph{Trends In Logic XVI - Studia Logica International Conference}) 
\cite{trendsinlogic}.

Atualmente as pesquisas estão focadas no 
estudo da aplicação da lógica paraconsistente, 
e ganhar espaço no universo técnico e científico, 
contribuindo com uma nova e eficiente forma de trabalho.





%%%%%%%%%%%%%%%%%%%%%%%%%%%%%%%%%%%%%%%%%%%%%%%%%%%%%%%%%%%%
\section{Objetivo Geral}
%%%%%%%%%%%%%%%%%%%%%%%%%%%%%%%%%%%%%%%%%%%%%%%%%%%%%%%%%%%%
%Objeto, subdividido em:
%Material: aquilo que pretende estudar, analisar, interpretar ou verificar de modo geral.
O desenvolvimento de um controlador utilizando a LPA2v que atenda um dado
requisito de desempenho para um dado sistema, 
de forma a comparar com uma implementação utilizando 
como técnica de controle um PID.



%%%%%%%%%%%%%%%%%%%%%%%%%%%%%%%%%%%%%%%%%%%%%%%%%%%%%%%%%%%%
\section{Objetivos Específicos}
%%%%%%%%%%%%%%%%%%%%%%%%%%%%%%%%%%%%%%%%%%%%%%%%%%%%%%%%%%%%
%Formal: enfoque especial, em face de diversas ciências que possuem o mesmo objetivo material.

Montar um sistema físico de forma a poder realizar os testes 
de controle com os dois modelos propostos.

Utilizar um controlador de alto poder de processamento, 
com núcleo ARM e periféricos dedicados, ou seja, 
módulo de cálculo como a unidade lógica e aritmética e 
a unidade de cálculo em ponto flutuante.

Gerar um ambiente para trabalhar com o controlador escolhido em um ambiente baseado em software livre ou aberto, de forma a poder compilar o firmware, gravá-lo e gerar toda a documentação deste trabalho dentro dessa filosofia.

Realizar o estudo e a implementação do controlador do tipo PID.

Implementar o controlador utilizando LPA2v, 
gerar as funções de controle necessárias utilizando os preceitos da lógica, 
sendo que não há funções dedicadas à LPA2v nativas ao controlador.

Um enfoque especial é dado a implementação do controlador utilizando LPA2v 
em função da inovação proposta e 
sua verificação mesmo que em um nível ainda inicial de testes.



%%%%%%%%%%%%%%%%%%%%%%%%%%%%%%%%%%%%%%%%%%%%%%%%%%%%%%%%%%%%
\section{Justificativa}
%%%%%%%%%%%%%%%%%%%%%%%%%%%%%%%%%%%%%%%%%%%%%%%%%%%%%%%%%%%%

%Função: aperfeiçoamento, através de crescente acervo de conhecimento, da relação do homem com o seu mundo.
Sendo o presente trabalho uma 
análise comparativa entre 
um sistema de controle clássico e 
uma lógica moderna para implementar o sistema de controle, 
sua função primordial é 
contribuir para a ampliação do conhecimento em 
uma nova forma de lidar com o mundo, 
amparado por um saber fortemente enraizado que 
serve de suporte comparativo a potencial teoria emergente, 
contribuindo dentro de uma aspecto ainda pouco explorado, 
mesmo contando com algumas implementações, 
há escassez de comparações diretas entre 
as técnicas de controle aqui presentes.



%%%%%%%%%%%%%%%%%%%%%%%%%%%%%%%%%%%%%%%%%%%%%%%%%%%%%%%%%%%%
\section{Limitações da pesquisa}
%%%%%%%%%%%%%%%%%%%%%%%%%%%%%%%%%%%%%%%%%%%%%%%%%%%%%%%%%%%%

%%% Não colocar o elemento TEMPO! 
%%% Colocar até onde vai a pesquisa e o que ela não abordará.

Este trabalho apresenta dois modos de controle, 
sendo o primeiro um modelo clássico utilizando PID e 
um segundo modelo utilizando LPA2v, 
como controle moderno. 
Não é objetivo do trabalho se aprofundar em questões históricas e 
nem explorar as diversas técnicas de controle, 
sintonia ou implementação.

O controlador utilizado possui 
suporte para implementação de PID, 
mas não é objetivo mostrar como se configura, 
ou mesmo abordar lógica de programação e seus fundamentos.



%%%%%%%%%%%%%%%%%%%%%%%%%%%%%%%%%%%%%%%%%%%%%%%%%%%%%%%%%%%%
\section{Estrutura do Trabalho}
%%%%%%%%%%%%%%%%%%%%%%%%%%%%%%%%%%%%%%%%%%%%%%%%%%%%%%%%%%%%

Este estudo apresenta em seu Capítulo 1, 
uma \textbf{Introdução} inicial ao tema de estudo, 
para uma ambientação da questão estudada. 
Na sequência, o Capítulo 2 apresenta o 
\textbf{Sistema eletrônico} 
utilizado como planta de implementação e 
testes para o estudo aqui apresentado, 
numa abordagem simples e direta, 
pois não se trata do objetivo principal deste estudo. 
O Capítulo 3, 
\textbf{Controle de Sistemas} contém o núcleo teórico, 
abordando ferramentas como o diagrama de blocos, 
o controle clássico, 
as ações de controle, 
requisitos de desempenho do sistema,
o controle não convencional utilizando a LPA2v,
já introduz alguns resultados 
obtidos utilizando o sistema mostrado no capítulo anterior. 
O Capítulo 4 é a 
\textbf{Apresentação e discussão dos resultados} 
onde é mostrada a implementação do controlador, 
o diagrama de blocos e código desenvolvidos, 
assim como gráficos com resultados alcançados.
Por fim o Capítulo 5, 
\textbf{Considerações Finais - Conclusão} 
é feito um fechamento com considerações sobre os objetivos propostos e alcançados e sobre a perspectiva para trabalhos futuros.


