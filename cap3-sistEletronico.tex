\section{Controlador}
\section{Drive}
\section{Aquecedor}
\subsection{ Temperatura }
A temperatura regula diversas propriedades físicas, químicas e biológicas como o caso do corpo humano, reações catalíticas, maturação fetal. Pode ser definida tanto no nível macroscópico, pela sensação de quente e frio, quanto no nível microscópico através do grau de agitação das moléculas. É uma propriedade intensiva de um sistema, ou seja, não depende da massa, assim como a viscosidade, pressão e densidade, contrapondo-se à propriedade extensiva como a própria massa, o volume, a energia cinética e a quantidade de movimento.
A temperatura governa o processo de transferência de energia térmica, calor, entre os corpos. Assim diz-se que quando há um equilíbrio térmico, há uma igualdade de temperatura.

\subsection{ Unidade de Temperatura }
O Sistema Internacional de Unidades através da 10ª Comissão Geral de Pesos e Medidas (CGPM, 1954) define a unidade de temperatura como sendo $\frac{1}{273,16} $  do ponto triplo da água\footnote{Ponto de equilíbrio entre temperatura, pressão e volume aonde a água pode coexistir nos estados sólido, líquido e gasoso.}. E na 13ª CGPM adota o nome kelvin, e K como símbolo padrão para temperatura.
Isso significa que a temperatura do ponto tríplo da água é:
\[ 273,16 \ kelvins \   ou T_{tpw} = 273,16 K \]

Como modo prático, é adotado o símbolo T e $ T_0 = 273.15K $ como temperatura de fusão da água. Assim para um dado valor de temperatura na escala kelvin tem-se que:
\[t = T - T_0\]
Sendo \emph{t} a temperatura em graus Celsius, com símbolo \textcelsius. Ambas as unidades de medida de temperatura, kelvin e graus Celsius, são unidades da Escala Internacional de Temperatura (International Temperature Scale) de 1990.
A conversão entre ambas as unidades é:
\[ t[ ^o C ] = T[K] - 273,15 \]

Ambas as escalas são ditas centígradas por possuírem cem divisões entre seus pontos de referência, fusão e ebulição da água.

Os pontos de referência são adotados a partir de condições estáveis de determinados sistemas, ou substâncias, para garantir uma uniformidade e a possibilidade de aferição dos instrumentos de medição.



\section{Sensores}
\section{Programação}
\subsection{Linguagem}
\subsection{Compilador}
\subsection{Técnicas de Programação}

