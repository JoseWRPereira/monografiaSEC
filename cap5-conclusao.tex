Este trabalho caracterizou de forma simples e expositiva a 
implementação dos diversos tipos de controlador, 
desde o liga-desliga até o PID completo, 
com cada respectiva codificação e 
resultado aquisitado diretamente na planta. 

Um modelo matemático foi desenvolvido, 
a análise de erro para validar seu uso como referência e assim
gerar um critério de orientação para os requisitos de desempenho do sistema, 
e para a comparação entre os sistemas,
clássico e não convencional, proposta pelo autor.

A LPA2v foi caracterizada desde sua abordagem básica, 
sua construção foi apresentada a forma mais usual, 
citando algumas aplicações já realizadas, e por fim, 
foi mostrado como a LPA2v foi utilzada para o controle dinâmico
 de um motor DC através de acionamento por PWM.

O fato de utilizar um hardware suficientemente capaz de 
processar dados em ponto flutuante 
permitiu a implementação direta da LPA2v, 
sem haver a necessidade da mudança de escala, 
bem como sua capacidade de processamento permitiu uma ótima resolução, 
com intervalos de aquisição de dados com intervalos de 10ms, 
independente da leitura do sensor de velocidade que ocorre utilizando 
interrupção.

Apesar do ótimo resultado alcançado, 
pois atendeu plenamente os requisitos propostos, 
e a implementação do controlador utilizando a LPA2v
se mostrar simples na codificação, 
ainda há uma grande gama de possibilidades a serem estudas e 
nuances da lógica a serem explorados em trabalhos futuros.


Este trabalho foi desenvolvido em sua totalidade 
utilizando a filosofia do software livre, 
e assim podem ser listadas as seguintes aplicações:

\begin{itemize}
\item Sistema Operacional: Debian GNU/Linux 8 (Jessie);
\item Shell: GNOME Shell 3.14.4;
\item Editor (texto e código): VIM - Vi IMproved 7.4;
\item Compilador: (GCC for ARM) arm-none-eabi-gcc (v.4.8.4-1+11-1);
\item Configuração de compilação: GNU Make 4.0;
\item Processador de Texto: \LaTeX  - pdfTex 3.14159265-2.6-1.40.15;
\item Gerador de Figuras: \LaTeX - Pacotes Ti$k$Z e PGF
\item Gerador de Figuras: GNU pic (groff)  version 1.22.2
\item Gerador de Gráficos: gnuplot 4.6 patchlevel 6
\item Terminal de Comunicação: minicom version 2.7 (compilado Jan 1 2014);
\item Gravador: LM4Flash version 0.1.3 - Flash for Stellaris Launchpad ICDI boards;
\end{itemize} 

O projeto em sua integralidade pode ser encontrado no GitHub:

Monografia: \url{https://github.com/JoseWRPereira/monografiaSEC.git}

Firmware: \url{https://github.com/JoseWRPereira/dcmotorcontrol.git}






%\citacao{
% typesetting systems: \LaTeX{}
%}}
